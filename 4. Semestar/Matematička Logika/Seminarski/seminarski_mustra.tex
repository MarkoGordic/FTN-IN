\documentclass[12pt, a4paper, titlepage, twoside]{article}
\usepackage{graphicx}
\usepackage{amssymb, amsfonts, amsmath, amsthm}
\usepackage[margin=3cm]{geometry}
\usepackage{forest}
%\usepackage[linguistics]{forest}
\usepackage{sectsty}
\usepackage{graphicx}

\sectionfont{\fontsize{16}{15}\selectfont}

\renewcommand{\refname}{Literatura}

\renewcommand{\contentsname}{Sadr\v zaj}

\newtheorem{theorem}{Teorema}[section]
\newtheorem{lemma}[theorem]{Lema}

\theoremstyle{definition}

\newtheorem{defn}[theorem]{Definicija}
\newtheorem{pr}{\sc Primer}


\renewcommand*{\proofname}{Dokaz}


\def\dj{d\kern-0.4em\char"16\kern-0.1em}
\def\Dj{\mbox{\raise0.3ex\hbox{-}\kern-0.4em D}}
\def\zn{,\kern-0.09em,}


\author{XXX YYY}

\begin{document}


\begin{titlepage}


\begin{center}
%\includegraphics[height=2cm]{logoUUNS}
\includegraphics[height=2cm]{uns_ftn_logo}
\end{center}

\begin{center}
\large UNIVERZITET U NOVOM SADU \\
\large FAKULTET TEHNI\v CNIH NAUKA\\[3.5cm]
\end{center} 


\begin{flushleft}
\large XXX YYY\\[3.5cm]
\end{flushleft}

\begin{center}
\huge  Matemati\v cka logika\\
%\huge  Metod rezolucije u iskaznoj logici\\
\large - seminarski rad - \\[3.5cm]
\end{center}

\begin{flushleft}
\large Profesor: dr Silvia Gilezan \\
\large Asistent: Simona Ka\v sterovi\'c \\[4.5cm]
\end{flushleft}

\begin{center}
\large Novi Sad, 2022.
\end{center}

\end{titlepage}

\tableofcontents

\newpage


\section{Uvod}
\vspace{1cm} \hspace{0.5cm}
Ponekad je te\v sko prona\'ci dokazni niz u iskaznom ra\v cunu za neku formulu, 
jer dokazni niz se \zn ne vidi$"$ iz same formule koju treba
dokazati, nego se \v cesto treba ne\v cega \zn dosetiti$"$. Prema tome, nije lako sprovesti
na ra\v cunaru traganje za dokazom u Hilbertovom deduktivnom
sistemu. \emph{Metod rezolucije} je postupak za dokazivanje da li je neka iskazna
(ili predikatska) formula zadovoljiva, a koji se mo\v ze lak\v se sprovesti na
ra\v cunaru. Kako je formula tautologija (odnosno valjana) akko njena negacija
nije zadovoljiva, metod rezolucije se mo\v ze koristiti za dokazivanje da li je neka
formula tautologija (odnosno da li je valjana). Tako\dj{}e, kako je formula B
semanti\v cka posledica kona\v cnog skupa formula {$A_1, A_2, \dots , A_n$} akko je formula 
$A_1 \land A_2 \land \dots \land A_n \Rightarrow B$ 
tautologija (odnosno valjana), metod rezolucije
je primenljiv i za ispitivanje da li je neka formula B semanti\v cka posledica
nekog kona\v cnog skupa hipoteza.


Metod rezolucije formulisao je \emph{Alan Robinson\footnote{D\v zon Alan Robinson 
(John Alan Robinson) (9. mart 1930. - 5. avgust 2016.) bio je filozof, matemati\v car 
i kompjuterski nau\v cnik. Bio je profesor na Univerzitetu u Sirakuzi. }} 1965. godine (u radu
pod naslovom \zn A machine oriented logic based on the resolution principle$"$), oslanjaju\'ci se na 
mnogobrojne rezultate koji su prethodili tom otkri\'cu. 

\newpage
\section{Konjunktivna normalna forma i pravilo rezolucije}
\vspace{1cm}\hspace{0.5cm}
Metod rezolucije primenjuje se na iskazne formule koje su u konjunktivnoj
normalnoj formi, tj. za formule koje su konjunkcije nekih disjunkcija. Za svaku iskaznu 
formulu postoji njoj ekvivalentna formula koja je u konjunktivnoj normalnoj formi. 


Algoritam za dobijanje konjunktivne normalne forme neke iskazne formule
dakle ima slede\'ce korake:
\begin{itemize}
	\item eliminisati veznike $\Rightarrow$ i $\Leftrightarrow$, kori\v s\'cenjem relacija $A \Rightarrow B \equiv \neg A \lor B$ i $A \Leftrightarrow B \equiv (A \Rightarrow B) \land (B \Rightarrow A)$
	\item kori\v s\'cenjem De Morganovih zakona \emph{\zn gurati negacije dublje"} tj. relacije $\neg (A \land B) \equiv (\neg A \lor \neg B)$ ili $\neg (A \lor B) \equiv (\neg A \land \neg B)$ koristiti \emph{s leva na desno}
	\item eliminisati duple negacije $\neg \neg A \equiv A$
	\item koristiti zakon distributivnosti $A \lor (B \land C) \equiv (A \lor B) \land (A \lor C)$
\end{itemize}


Pretpostavimo da je formula koju \v zelimo ispitati u konjunktivnoj normalnoj formi. Dakle, 
formula je konjunkcija nekih disjunkcija, i to disjunkcija nekih iskaznih slova ili 
negacija iskaznih slova. Rezolucijsko pravilo mo\v ze se primeniri samo na disjunkcije, 
pa premise trebaju biti u obliku disjunkcije tj. trebaju biti klauze.


\begin{defn}
Neka je $S$ skup iskaznih slova. Literal je svako iskazno slovo ili negacija iskaznog slova, tj. 
svaki $p$ odnosno $\neg p$, gde $p \in S$. Literal suprotan literalu $p$ jeste literal $\neg p$, 
i obrnuto.
\end{defn}


\begin{defn}
Klauza je disjunkcija literala. Prazna klauza, u oznaci $\square$, je klauza koja ne sadr\v zi 
nijedan literal.
\end{defn}


Zbog asocijativnosti i komutativnosti konjunkcije i disjunkcije,
formulu koja je u konjunktivnoj normalnoj formi mo\v zemo smatrati multiskupom
klauza, pri \v cemu svaku klauzu mo\v zemo smatrati multiskupom literala.
Na osnovu logi\v ckih ekvivalencija $A \land A \equiv A$ i $A \lor A \equiv A$
mogu se eliminisati
vi\v sestruka pojavljivanja jedne klauze u formuli i vi\v sestruka pojavljivanja
jednog literala u klauzi. Time formula kao multiskup klauza od kojih je svaka
multiskup literala mo\v ze da se zameni (logi\v cki ekvivalentnom) formulom koja
je skup klauza od kojih je svaka skup literala. Metod rezolucije ispituje da li je
zadat skup klauza zadovoljiv.


Klauza je zadovoljiva ako postoji valuacija u kojoj je bar jedan literal iz
te klauze ta\v can. Klauza je pobijena valuacijom u kojoj su svi literali neta\v cni.
Prazna klauza, u oznaci $\square$ , ne sadr\v zi nijedan literal i nije zadovoljiva. Formula
koja je skup klauza je zadovoljiva ako postoji valuacija u kojoj su sve klauze te
formule ta\v cne, a ina\v ce je nezadovoljiva.


Jednostavnosti radi, sve klauze koje sadr\v ze logi\v cke konstante $\top$ ili $\bot$ mogu
biti eliminisane ili zamenjene tako da se ne promeni zadovoljivost polaznog
skupa klauza. Klauza koja sadr\v zi literal $\top$ je u svakoj valuaciji ta\v cna, pa mo\v ze 
biti eliminisana (jer ne uti\v ce na zadovoljivost polaznog skupa klauza). Ako
klauza $C$ sadr\v zi literal $\bot$, onda taj literal mo\v ze biti obrisan, daju\'ci novu klauzu
$C'$ (jer je u svakoj valuaciji klauza $C$ ta\v cna ako i samo ako je ta\v cna klauza $C'$).
Dakle, mo\v ze se smatrati da je u svim klauzama svaki literal ili iskazno slovo
ili negacija iskaznog slova. Ako je literal $l$ jednak iskaznom slovu $p$, onda
sa $\overline{l}$ ozna\v cavamo literal $\neg p$; ako je literal $l$ jednak negaciji iskaznog 
slova $p$ (tj. literalu $ \neg p $), onda sa $\overline{l}$ ozna\v cavamo literal $p$. Za 
literale $l$ i $\overline{l}$ ka\v zemo da su me\dj{}usobno komplementni.


Ako su $C'$ i $C"$ klauze, onda se pravilom rezolucije iz klauza $C' \lor l$ i 
$C" \lor \overline{l}$ izvodi klauza $C' \lor C"$. Pravilo rezolucije mo\v ze se prikazati 
u slede\'cem obliku:
$$\frac{C' \lor l \quad C" \lor \overline{l}}{C' \lor C"}$$
Klauzu $C' \lor C"$ zovemo rezolventom klauza $C' \lor l$ i $C" \lor \overline{l}$,
a klauze $C' \lor l$ i $C" \lor \overline{l}$ roditeljima rezolvente. Ka\v zemo da klauze 
$C' \lor l$ i $C" \lor \overline{l}$ rezolviramo pravilom rezolucije.


Pravilo rezolucije intuitivno mo\v ze biti shva\'ceno na slede\'ci na\v cin: klauza $C' \lor p$ 
logi\v cki je ekvivalentna formuli $\neg C' \Rightarrow p$, klauza $C" \lor \neg p$ 
logi\v cki je ekvivalentna formuli $p \Rightarrow C"$, a formule $\neg C' \Rightarrow p$ i 
$p \Rightarrow C"$ imaju logi\v cku posledicu $\neg C' \Rightarrow C"$ koja je logi\v cki 
ekvivalentna klauzi $C' \lor C"$.


Ako je dat skup klauza $S$, pravilom rezolucije se roditelji rezolvente ne zamenjuju
rezolventom, ve\'c se rezolventa dodaje u skup $S$.


Metod rezolucije je postupak za ispitivanje zadovoljivosti skupa klauza koji
se sastoji od uzastopnog primenjivanja pravila rezolucije, tj. metod rezolucije se sastoji 
od niza primena pravila rezolucije na skupu klauza. Neka je $S$ po\v cetni
skup, neka je $S_0 = S$ i neka je $S_{i+1}$ rezultat primene pravila rezolucije na skup
$S_i$. Postupak se zaustavlja na jedan od slede\'ca dva na\v cina: 
\begin{itemize}
	\item ako u nekom koraku skup $S_i$ sadr\v zi praznu klauzu ($\square$ ), onda zaustavi
primenu procedure i vrati odgovor da je skup klauza $S$ nezadovoljiv;
	\item ako ne postoji mogu\'cnost da se primeni pravilo rezolucije tako da se
skupovi $S_i$ i $S_{i+1}$ razlikuju, onda zaustavi primenu procedure i vrati
odgovor da je skup klauza $S$ zadovoljiv.
\end{itemize}


Niz klauza (polaznih i izvedenih) ozna\v cava\'cemo obi\v cno sa $C_i$ ($i = 1, 2,\dots$).
Iza izvedene klauze zapisiva\'cemo oznake klauza iz kojih je ona izvedena, kao
i redne brojeve literala nad kojim je primenjeno pravilo rezolucije. Literale u
klauzama razdvaja\'cemo obi\v cno simbolom \zn ~, $"$ (umesto simbolom \zn $ \lor $ $"$).


\begin{pr}
Metodom rezolucije se iz skupa \{\{$\neg p, \neg q, r$\}, \{$\neg p, q$\}, \{$p$\}, \{$\neg r$\}\}
mo\v ze izvesti prazna klauza:
\begin{center}
\begin{tabular}{l l l}
	
	$C_1 :$ & $\neg p, \neg q, r$  \\
	
	$C_2 :$ & $\neg p, q$  \\
	
	$C_3 :$ & $p$  \\
	
	$C_4 :$ & $\neg r$ \\ \hline
	
	$C_5 :$ & $\neg p, r$ & $(C1, 2; C2, 2)$ \\
	
	$C_6 :$ & $\neg p$ & $(C4, 1; C5, 2)$ \\
	
	$C_7 :$ & $\square$ &  $(C3, 1; C6, 1)$ 
	
\end{tabular}
\end{center}
Skup klauza \{\{$\neg p, \neg q, r$\}, \{$\neg p, q$\}, \{$p$\}, \{$\neg r$\}\} je, 
dakle, nezadovoljiv.
\end{pr}


\begin{pr}
Metodom rezolucije se iz skupa $\{\{\neg p, \neg q, r\}, \{\neg p, q\}, \{p\}\}$ ne mo\v ze
izvesti prazna klauza. Ovaj skup klauza je, dakle, zadovoljiv.
\begin{center}
\begin{tabular}{l l l}
	
	$C_1 :$ & $\neg p, \neg q, r$  \\
	
	$C_2 :$ & $\neg p, q$  \\
	
	$C_3 :$ & $p$  \\ \hline
	
	$C_4 :$ & $\neg p, r$ & $(C1, 2; C2, 2)$ \\
	
	$C_5 :$ & $r$ & $(C3, 1; C4, 1)$ \\

\end{tabular}
\end{center}

\end{pr}

\vspace{0.2cm}
\begin{theorem}[Zaustavljanje metoda rezolucije]
Metod rezolucije se zaustavlja.
\end{theorem}

\vspace{0.2cm}
\begin{proof}
U svakom koraku metoda rezolucije teku\'cem skupu se dodaje nova
klauza.

\noindent
Klauze su skupovi literala, pa nije bitan poredak literala u njima, a vi\v sestruka 
pojavljivanja literala u klauzi nisu mogu\'ca. Nad skupom od $n$
iskaznih slova ima $2n$ razli\v citih literala (za svako iskazno slovo $p$ postoje
literali $p$ i $\neg p$) i svaki od njih mo\v ze da se pojavljuje ili ne pojavljuje u
{klauzi\footnote{Ukoliko bi bilo uvedeno ograni\v cenje da klauzula ne mo\v ze da sadr\v zi 
i neko iskazno slovo i njegovu negaciju (takva klauzula je u svakoj valuaciji ta\v cna), 
onda bi nad $n$ iskaznih slova bilo $3^n$ razli\v citih klauzula (jer svako iskazno slovo 
$p$ mo\v ze da se ne pojavljuje u klauzuli, da se pojavljuje u vidu literala $p$ ili 
da se pojavljuje u okviru literala $\neg p$).}}, te razli\v citih klauza ima $2^{2n} = 4^n$.

\noindent
Dakle, metod rezolucije se zaustavlja u kona\v cno mnogo koraka, jer se iz
literala iz skupa $S$ mo\v ze izvesti samo kona\v can broj (novih) klauza. (Taj
broj me\dj{}utim mo\v ze biti eksponencijalno veliki (u funkciji broja literala u
$S$) i primena procedure rezolucije mo\v ze da se sastoji od veoma velikog
broja koraka.) 
\end{proof}

\vspace{0.2cm}
\begin{theorem}[Saglasnost pravila rezolucije]
Neka je skup klauza $S'$ dobijen od skupa klauza $S$ primenom pravila rezolucije. 
Ako neka valuacija zadovoljava skup $S$, onda ona zadovoljava i skup $S'$.
\end{theorem}

\vspace{0.2cm}
\begin{proof}
Pretpostavimo da valuacija $v$ zadovoljava sve klauze iz skupa $S$. Doka\v zimo da ona 
zadovoljava i sve klauze iz skupa $S'$. 

\noindent
Pretpostavimo da je
pravilo rezolucije primenjeno na klauze $C_a$ i $C_b$ i da je njihova rezolventa
klauza $C_r$. Jedina klauza koja pripada skupu $S'$, a mogu\'ce ne pripada
skupu $S$ je klauza $C_r$. Doka\v zimo da valuacija $v$ zadovoljava klauzu $C_r$.

\noindent
Pretpostavimo da klauza $C_a$ sadr\v zi literal $l$, a klauza $C_b$ literal $\overline{l}$. Ako je
literal $l$ ta\v can u valuaciji $v$, onda je literal $\overline{l}$ neta\v can, pa u 
klauzi $C_b$ mora
da postoji neki literal (razli\v cit od $\overline{l}$) koji je ta\v can u valuaciji $v$. Taj literal
je i element klauze $C_r$, pa je i ona zadovoljiva u valuaciji $v$. Ako je literal
$l$ neta\v can u valuaciji $v$, onda u klauzi $C_a$ mora da postoji neki literal 
(razli\v cit od $l$) koji je ta\v can u valuaciji $v$. Taj literal je i element klauze $C_r$, 
pa je i ona zadovoljiva u valuaciji $v$. 

\noindent
Dakle, svaka klauza iz skupa $S'$ je ta\v cna
u valuaciji $v$, odakle sledi tvr\dj{}enje teoreme.
\end{proof}

\vspace{0.2cm}
\begin{theorem}[Saglasnost metoda rezolucije]
Ako se iz skupa klauza $S$ mo\v ze izvesti prazna klauza, onda je $S$ nezadovoljiv skup klauza.
\end{theorem}

\vspace{0.2cm}
\begin{proof}
Na osnovu teoreme saglasnost pravila rezolucije (i na osnovu jednostavnog induktivnog argumenta),
ako je skup klauza $S$ zadovoljiv, zadovoljiv je i svaki skup klauza
$S'$, dobijen u nekoj iteraciji metoda. Obratno, skup klauza $S$ je nezadovoljiv
ako je nezadovoljiv skup $S'$, dobijen u nekoj iteraciji metoda. 

\noindent
Dakle,mako metod rezolucije u nekom koraku doda praznu klauzu u teku\'ci skup
klauza, sledi da je skup $S$ nezadovoljiv, \v sto je i trebalo dokazati.
\end{proof}
\vspace{0.2cm}

Teorema o saglasnosti rezolucije tvrdi da iz zadovoljivog skupa ne mo\v ze
biti izvedena prazna klauza. Pokaza\'cemo u nastavku da iz svakog nezadovoljivog
skupa klauza mo\v ze biti izvedena prazna klauza. Razmatra\'cemo stablo
sa mogu\'cim va\-lua\-ci\-ja\-ma fiksnog skupa iskaznih slova ${p_1, p_2,\dots , p_m}$.
Zato najpre navodimo definiciju stabla.

\vspace{1cm}
\section{Stablo}
\vspace{1cm}
Rezolucijsko izvo\dj{}enje se ponekad prikazuje i u formi stabla.
\begin{defn}
Neure\dj{}eno stablo $T$ je par $(S, \prec)$ sa slede\'cim svojstvima:
\begin{itemize}
	\item $S$ je skup \v cije elemente zovemo \v cvorovima neure\dj{}enog stabla;
	\item $\prec$ je binarna relacija nad $S$, $x \prec y$ \v citamo \v cvor $x$ je prethodnik
(roditelj, direktni predak) \v cvora $y$ ili \v cvor $y$ je sledbenik (dete, direktni potomak)
\v cvora $x$; ova relacija zadovoljava slede\'ce uslove:
\begin{itemize}
	\item[-] postoji jedinstven \v cvor koji nema prethodnika, taj \v cvor zovemo koren
stabla;
	\item[-] svaki \v cvor razli\v cit od korena ima ta\v cno jednog prethodnika;
	\item[-] ne postoji niz \v cvor ova $x_1, x_2, \dots , x_n$ takav da va\v zi 
$x_1 \prec x_2 \prec \dots \prec x_n \prec x_1$.
\end{itemize}
\end{itemize}


Ako za niz \v cvorova $x_1, x_2,\dots, x_n$ va\v zi 
$x_1 \prec x_2 \prec \dots \prec x_{n-1} \prec x_n$, 
onda ka\v zemo da je \v cvor $x_1$ predak \v cvora $x_n$ i da je \v cvor $x_n$ 
potomak \v cvora $x_1$.


List stabla je  \v cvor koji nema sledbenika. Staza stabla (grana stabla) je 
niz \v cvorova $x_1, x_2, \dots , x_n$, pri  \v cemu je $x_1$ koren stabla i  \v cvor $x_1$ je 
prethodnik \v cvora $x_2$, \v cvor $x_2$ je prethodnik \v cvora $x_3,\dots,x_{n-1}$ je 
prethodnik  \v cvora $x_n$. Maksimalna staza (maksimalna grana) je niz  \v cvorova 
$x_1, x_2,\dots , x_n$, pri  \v cemu je $x_1$ koren stabla, $x_n$ je list stabla 
i  \v cvor $x_1$ je prethodnik \v cvora $x_2$, \v cvor $x_2$ je prethodnik  \v cvora 
$x_3,\dots,x_{n-1}$ je prethodnik \v cvora $x_n$.


Ako je $x_1$ koren stabla i ako postoji niz  \v cvorova $x_1, x_2,\dots,x_k$ takav da va\v zi
$x_1 \prec x_2 \prec \dots \prec x_{k-1} \prec x_k$, onda ka\v zemo da je visina  
\v cvora $x_k$ jednaka $k$.


Ako  \v cvor stabla ima (samo) direktne potomke $x_1, x_2, \dots , x_k$ ($k \geq 0$), onda
ka\v zemo da je stepen tog  \v cvora jednak $k$.
\end{defn}


\begin{defn}
Za stablo ka\v zemo da je binarno ako svaki \v cvor ima najvi\v se dva
sledbenika. Za stablo ka\v zemo da je potpuno binarno ako postoji vrednost $k$
takva da svaki \v cvor visine manje ili jednake $k$ ima po ta\v cno dva sledbenika i
nijedan \v cvor visine $k + 1$ nema nijednog sledbenika.
\end{defn}


\begin{defn}
Ure\dj{}eno stablo je neure\dj{}eno stablo za \v ciji je svaki \v cvor skup
njegovih sledbenika ure\dj{}en. 
\end{defn}


\begin{center}
\begin{forest}
[$\epsilon$
	[$p_1$
		[$p_1 p_2$
			[$p_1 p_2 p_3$]
			[$p_1 p_2 \overline{p_3}$]
		]
		[$p_1 \overline{p_2}$
			[$p_1 \overline{p_2} p_3$]
			[$p_1 \overline{{p}_2 {p}_3}$]
		]
	]
	[$\overline{p_1}$
		[$\overline{p_1} p_2$
			[$\overline{p_1} p_2 p_3$]
		   	[$\overline{p_1} p_2 \overline{p_3}$]
		]
		[$\overline{{p}_1 {p}_2}$
			[$\overline{{p}_1 {p}_2} p_3$]
		   	[$\overline{{p}_1 {p}_2 {p}_3}$]
		]
	]
]
\end{forest}		\\					
\vspace{0.2cm}
Slika 1: Stablo valuacije za tri iskazna slova				
\end{center}


Stablo valuacija je ure\dj{}eno potpuno binarno stablo visine $m$, \v cijem je svakom
\v cvoru pridru\v zeno dodeljivanje vrednosti $0$ ili $1$ iskaznim slovima iz zadatog
skupa i dodeljivanje za svaki \v cvor (sem listova) je u njegovim direktnim potomcima
pro\v sireno dodeljivanjem vrednosti $0$ i $1$ jo\v s jednom iskaznom slovu.
U korenu stabla je prazna dodela (ozna\v cena sa $\epsilon$) koja ne pridru\v zuje vrednost
nijednom iskaznom slovu. \v Cvoru stablu valuacija odgovara parcijalnoj dodeli
vrednosti $0$ i $1$ iskaznim slovima iz nekog skupa. Parcijalne dodele mogu biti
dovoljne za utvr\dj{}ivanje istinitosne vrednosti neke formule. Mo\v ze se govoriti
o parcijalnim dodelama koje pro\v siruju druge dodele. Parcijalna dodela koja
odgovara nekom \v cvoru stabla valuacije je pro\v sirenje svake dodele koje odgovara
prethodnicima tog \v cvora. Stablo valuacija ima $2^m$ listova, po jedan za
svaku od mogu\'cih valuacija. Parcijalne valuacije zapisujemo kratko kao niske
znakova iz skupa znakova od $p_1$ do $p_m$ i od $\overline{p_1}$ do $\overline{p_m}$. 
Sa $p$ ozna\v cavamo da valuacija iskaznom slovu $p$ dodeljuje vrednost $1$, a sa $\overline{p}$ 
da valuacija iskaznom slovu $p$ dodeljuje vrednost $0$. Slika 1 ilustruje stablo valuacija za 
tri iskazna slova.


Neka je $S$ skup klauza i neka je $S_k$ poslednji skup klauza dobijen metodom
rezolucije. Klauza $C$ pokriva \v cvor $n$ u stablu valuacija ako ona zadovoljava
slede\'ce uslove:
\begin{itemize}
	\item klauza $C$ je element skupa $S_k$;
	\item iskazna slova koja se pojavljuju u $C$ su me\dj{}u slovima kojima su dodeljene
vrednosti u \v cvoru $n$;
	\item klauza $C$ je neta\v cna u valuaciji koja odgovara \v cvoru $n$.
\end{itemize}


Na osnovu drugog i tre\'ceg uslova sledi da je klauza $C$ koja pokriva \v cvor
$n$ neta\v cna i u svakoj valuaciji koja pro\v siruje valuaciju koja odgovara \v cvoru $n$
(tj. neta\v cna i u svakoj valuaciji koja odgovara nekom potomku \v cvora $n$). Neki
\v cvorovi mogu biti pokriveni od strane vi\v se klauza, dok neki ne moraju biti
pokriveni. Prazna klauza je jedina klauza koja mo\v ze pokriti koren stabla valuacije.
Dodatno, prazna klauza mo\v ze pokriti bilo koji \v cvor stabla valuacije.

\vspace{0.2cm}
\begin{lemma}
Ako su, za skup klauza dobijen metodom rezolucije iz skupa klauza
$S$, oba deteta \v cvora stabla valuacija pokrivena, onda je i taj \v cvor pokriven.
\end{lemma}

\vspace{0.2cm}
\begin{proof}
Pretpostavimo da je $n$ \v cvor \v cija su oba deteta pokrivena i neka je $v$ valuacija
koja mu je pridru\v zena. Neka su \v cvorovi $n_1$ i $n_2$ dva deteta \v cvora $n$,
neka su im pridru\v zene valuacije $vp$ i $v\overline{p}$ i neka su $C_1$ i $C_2$ klauze koje,
redom, pokrivaju ta dva \v cvora. Ako se iskazno slovo $p$ ne pojavljuje u
$C_1$, onda $C_1$ pokriva \v cvor $n$. Sli\v cno, ako se iskazno slovo $p$ ne pojavljuje
u $C_2$, onda $C_2$ pokriva \v cvor $n$. Ako se iskazno slovo $p$ pojavljuje i u $C_1$
i u $C_2$, onda $C_1$ sadr\v zi literal $\neg p$, a klauza $C_2$ literal $p$. Zaista, \v cvoru 
$n_1$ odgovara valuacija koja dodeljuje iskaznom slovu $p$ vrednost $1$, pa literal
$p$ ne mo\v ze da se pojavljuje u $C_1$ (jer bi klauza $C_1$ tada bila ta\v cna u valuaciji
koja odgovara \v cvoru $n_1$). Analogno, klauza $C_2$ sadr\v zi literal $p$. Na klauze
$C_1$ i $C_2$ se, dakle, mo\v ze primeniti pravilo rezolucije (po iskaznom slovu
$p$). Neka je $C_r$ rezolventa ove dve klauze (po iskaznom slovu $p$). Ukoliko je iz skupa $S$ 
izvedena prazna klauza, onda je ona klauza koja
pokriva \v cvor $n$. Doka\v zimo da, ako prazna klauza nije izvedena, onda
klauza $C_r$ pokriva \v cvor $n$. Potrebno je dokazati tri svojstva klauze $C_r$.
\begin{itemize}
	\item Ako, kao \v sto je pretpostavljeno, prazna klauza nije izvedena iz $S$,
onda skup $S_k$ sadr\v zi sve mogu\'ce rezolvente izvedene iz skupa $S$, pa
je i klauza $C_r$ element skupa $S_k$.
	\item Iskazna slova koja se pojavljuju u klauzi $C_r$ su slova koja se pojavljuju
u klauzi $C_1$ ili klauzi $C_2$. Za svako slovo koje se pojavljuje u
ovim dvema klauzama va\v zi ili da mu je dodeljena vrednost u \v cvoru
$n$ ili da je ono upravo iskazno slovo $p$. Dovoljno je, dakle, pokazati
da iskazno slovo $p$ ne pripada klauzi $C_r$. Literal $p$ mo\v ze da pripada
klauzi $C_r$ samo ako pripada i klauzi $C_1$. Me\dj{}utim, ako bi literal
$p$ pripadao klauzi $C_1$, ta klauza bi sadr\v zala literale $p$ i $\neg p$, pa bi
bila ta\v cna u svakoj valuaciji, \v sto je u suprotnosti sa \v cinjenicom da
je klauza $C_1$ neta\v cna u valuaciji $vp$. Dakle, literal $p$ ne pripada klauzi
$C_1$ i, analogno, literal $\neg p$ ne pripada klauzi $C_2$. Odatle sledi da se
iskazno slovo $p$ ne pojavljuje u klauzi $C_r$, pa su sva slova koja se pojavljuju
u $C_r$ me\dj{}u slovima kojima su dodeljene vrednosti u \v cvoru
$n$.
	\item Klauza $C_1$ pokriva \v cvor $n_1$, pa svi njeni literali imaju vrednost $0$ u
valuaciji $vp$. Klauza $C_1$, kao \v sto je pokazano, ne sadr\v zi literal $p$, pa
je svaki literal iz skupa $C_1 \setminus \{\neg p\}$ neta\v can u valuaciji $v$. Analogno,
svaki literal iz skupa $C_2 \setminus \{p\}$ je neta\v can u valuaciji $v$. Dakle, svaki
literal klauze $C_r$ (koja je unija skupova $C_1 \setminus \{\neg p\}$ i $C_2 \setminus \{p\}$) 
je neta\v can u valuaciji $v$.
\end{itemize} \end{proof}


\vspace{0.2cm}
\begin{pr}
Primer 1 u obliku stabla.
\begin{center}
\begin{forest}
for tree={grow=north}
[$\square$
	[$\{p\}$]
	[$\{\neg p\}$
		[$\{\neg r\}$]
		[$\{\neg p ; r\}$
			[$\{\neg p; q\}$]
		   	[$\{\neg p ; \neg q ; r\}$]
		]
	]
]
\end{forest}	
\end{center}
\end{pr}

\vspace{1cm}
\section{Potpunost i teorema o metodu rezolucije}
\vspace{1cm}
\begin{theorem}[Potpunost metoda rezolucije]
Ako je $S$ nezadovoljiv skup klauza,
onda se iz njega mo\v ze izvesti prazna klauza.
\end{theorem}

\vspace{0.2cm}
\begin{proof}
Neka je $S_k$ poslednji skup dobijen metodom rezolucije od nezadovoljivog
skupa $S$. Neka je $T$ stablo valuacija za iskazna slova iz skupa $S$.
Kako je skup $S$ nezadovoljiv, on je neta\v can u svakoj valuaciji, tj. u svakoj
valuaciji $v$ bar jedna klauza iz $S$ je neta\v cna, te ona pokriva list stabla $T$
koji odgovara valuaciji $v$. Dakle, svaki list stabla $T$ je pokriven nekom
klauzom iz skupa $S$. Primenom leme 3.4 i na osnovu jednostavnog induktivnog
argumenta sledi da su svi \v cvorovi stabla $T$ pokriveni elementima
skupa $S_k$. Specijalno, i koren stabla $T$ mora biti pokriven, a prazna
klauza je jedina klauza koja mo\v ze da pokrije koren. Dakle, prazna klauza
je izvedena rezolucijom iz skupa $S$, \v sto je i trebalo dokazati.  
\end{proof}

\vspace{0.2cm}
Metod rezolucije se uvek zaustavlja (teorema 2.3), pa na osnovu dokaza
prethodne teoreme sledi da se iz svakog nezadovoljivog skupa klauza nu\v zno
mora izvesti prazna klauza bez obzira na izbor klauza za rezolviranje u pojedinim
koracima. Pa bi prethodna teorema glasila:


\zn Ako je $S$ nezadovoljiv skup klauza,
onda se iz njega mora izvesti prazna klauza.$"$ \\


Dodatno, iz toga sledi i da ukoliko za neki skup klauza
metod staje bez izvo\dj{}enja prazne klauze, onda je taj skup zadovoljiv.

\noindent
Na osnovu prethodnih teorema sledi naredno tvrdenje.

\vspace{0.2cm}
\begin{theorem}[Teorema o metodu rezolucije]
Metod rezolucije se zaustavlja za svaku iskaznu formulu i u zavr\v snom skupu klauza 
postoji prazna klauza ako i samo ako je polazna formula nezadovoljiva.
\end{theorem}

\newpage
Metod rezolucije mo\v ze biti modifikovan tako da bude efikasniji. Modifikacije
metoda mogu biti zasnovane na slede\'cim \v cinjenicama: 
\begin{itemize}
	\item ako je klauza $C$ tautologija, onda je skup $S$ zadovoljiv ako i samo ako je
skup $S \setminus \{C\}$ zadovoljiv;
	\item ako skup $S$ sadr\v zi jedini\v cnu klauzu \{$l$\}, onda je skup $S$ zadovoljiv ako
i samo ako je zadovoljiv skup dobijen od $S$ brisanjem svih klauza koje
sadr\v ze literal $l$ i, zatim, brisanjem svih pojavljivanja literala $\overline{l}$;
	\item ako se u skupu klauza $S$ pojavljuje literal $l$, a ne i literal $\overline{l}$, 
onda je skup $S$ zadovoljiv ako i samo ako je skup $T$ zadovoljiv, gde je skup $T$ skup svih
klauza iz $S$ koje ne sadr\v ze $l$;
	\item ako skup $S$ sadr\v zi klauze $C_0$ i $C_1$ i ako je klauza $C_0$ podskup klauze $C_1$,
onda je skup $S$ zadovoljiv ako i samo ako je zadovoljiv skup $S \setminus \{C_1\}$ (pravilo
subsumption).
\end{itemize}


Primetimo da su navedene \v cinjenice sli\v cne koracima {DPLL procedure\footnote
{Dejvis-Patnam-Logman-Loveladova procedura (DPLL procedura) vr\v si ispitivanje zadovoljivosti 
iskaznih formula. Ona se primenjuje na iskazne formule u konjunktivnoj normalnoj formi. 
Predstavljena je 1962. godine od strane Martina Dejvisa, Hilari Patnama, D\v zord\v za 
Logmana i Donalda Lovelanda i ona je usavr\v senje ranije Dejvis-Patnamove procedure, koja je 
razvijena od strane Dejvisa i Patnama 1960. godine.}}.
Za efikasnu primenu metoda rezolucije veoma je, u svakom koraku, bitan i
izbor para klauza nad kojima se primenjuje pravilo rezolucije.

\newpage
\section{Zaklju\v cak}
\vspace{1cm}\hspace{0.5cm}
U svom osnovnom obliku, metod rezolucije proverava da li je dati skup
klauza (ne)zadovoljiv. Me\dj{}utim, metod rezolucije mo\v ze se koristiti i za ispitivanje
va\-lja\-no\-sti. Naime, ako je potrebno ispitati da li je formula $A$ valjana,
dovoljno je metodom rezolucije utvrditi da li je formula $\neg A$ nezadovoljiva (pri
\v cemu je potrebno najpre formulu $\neg A$ transformisati u konjuktivnu normalnu
formu). Ovaj vid dokazivanja da je formula $A$ valjana zovemo dokazivanje pobijanjem.
Za metod rezolucije primenjen na ovaj na\v cin, saglasnost govori da
nije mogu\'ce rezolucijom pogre\v sno utvrditi (pobijanjem) da je neka formula valjana,
a potpunost govori da je za svaku valjanu formulu metodom rezolucije
mogu\'ce dokazati (pobijanjem) da je valjana.


Metodom rezolucije mo\v ze se ispitati i da li va\v zi $A_1, A_2,\dots , A_n \models B$. Ovo
tvr\dj{}enje je ta\v cno ako i samo ako je formula 
$A_1 \land A_2 \land \dots \land A_n \Rightarrow B$ valjana. 
Formula $A_1 \land A_2 \land \dots \land A_n \Rightarrow B$ je valjana 
ako i samo ako je formula $\neg (A_1 \land A_2 \land \dots \land A_n \Rightarrow B)$ 
nezadovoljiva, tj. ako i samo ako je formula
$\neg (\neg (A_1 \land A_2 \land \dots \land A_n) \lor B)$ tj. 
$A_1 \land A_2 \land\dots \land A_n \land \neg B$ nezadovoljiva. Ukoliko su formule 
$A_i$ u konjunktivnoj normalnoj formi, da bi na navedenu formulu bio primenjen metod rezolucije,
dovoljno je transformisati formulu $\neg B$ u konjunktivnu normalnu formu.


\begin{pr}
Va\v zi
$$(\neg p \lor \neg q \lor r) \land (\neg p \lor q) \land p \models r$$\\
ako i samo ako je skup klauza $\{\{\neg p, \neg q, r\}, \{\neg p, q\}, \{p\}, \{\neg r\}\}$ 
nezadovoljiv. U primeru 1 je pokazano da je taj skup klauza nezadovoljiv.
\end{pr}


Metoda rezolucije je po svojoj prirodi mehani\v cka, odnosno orijentisana ka ra\-\v cu\-nar\-skom
izvr\v savanju, \v sto se lako uo\v cava kod primene na ve\'ce skupove formula kada se pojavljuje 
ogroman broj klauza. \v Covek se tada te\v sko snalazi pretra\v zuju\'ci klauze koje treba 
rezolvirati. Naime, tokom rezolviranja od klauza roditelja se \v cesto dobijaju klauze koje nisu 
u polaznom skupu, niti su deo nekih klauza polaznog skupa, \v sto ima za posledicu potencijalno 
eksplozivni rast broja klauza. 

Deo materijala je preuzet iz \cite{gilezan2020logika}.


\newpage
\begin{thebibliography}{100}


\bibitem{1}
{J. A. Robinson.}
\newblock {\em A machine oriented logic based on the resolution principle}.
\newblock J. Assoc. Comput. Mach., 1965.

\bibitem{2}
{Jean H. Gallier.}
\newblock {\em Logic for computer science: Foundations of Automatic Theorem Proving.}.
\newblock New York, 1986.

\bibitem{gilezan2020logika}
{Silvia Gilezan, Simona Ka\v sterovi\'c.}
\newblock {\em Matemati\v cka logika, predavanja i ve\v zbe.}.
\newblock Fakultet tehni\v ckih nauka, Novi Sad, 2020.

\bibitem{3}
{Predrag Jani\v ci\'c.}
\newblock {\em Matemati\v cka logika u ra\v cunarstvu}.
\newblock Matemati\v cki fakultet, Beograd, 2008.

\bibitem{4}
{Rozalia Madaras Sila\dj{}i}
\newblock {\em Matemati\v cka logika}.
\newblock Prirodno-matemati\v cki fakultet, Novi Sad, 2012.

\bibitem{5}
{Gradimir Vojvodi\'c.}
\newblock {\em Predavanja iz matemati\v cke logike}.
\newblock Prirodno-matemati\v cki fakultet, Novi Sad, 2007.

\bibitem{6}
{Kosta Do\v sen}
\newblock {\em Osnovna logika}.
\newblock Beograd, 2013.

\bibitem{7}
{Zoran Ognjanovi\'c, Nenad Krd\v zavac}
\newblock {\em Uvod u teorijsko ra\v cunarstvo}.
\newblock Beograd - Kragujevac, 2004.

\end{thebibliography}

\end{document}